\documentclass[]{jsarticle}
\usepackage[dvipdfmx]{graphicx}
\usepackage{bm}
\usepackage{amsmath}
\usepackage{amssymb}
\usepackage{amsfonts}
\usepackage{amsthm}
\usepackage{comment}
\usepackage{listings}
\usepackage{empheq}
\usepackage{siunitx}
\usepackage[hyphens]{url}
\lstset{
    basicstyle={\ttfamily},
    identifierstyle={\small},
    commentstyle={\smallitshape},
    keywordstyle={\small\bfseries},
    ndkeywordstyle={\small},
    stringstyle={\small\ttfamily},
    frame={tb},
    breaklines=true,
    columns=[l]{fullflexible},
    numbers=left,
    xrightmargin=0zw,
    xleftmargin=3zw,
    numberstyle={\scriptsize},
    stepnumber=1,
    numbersep=1zw,
    lineskip=-0.5ex,
    keepspaces=true,
    language=c
}
\renewcommand{\lstlistingname}{リスト}
\makeatletter
\newcommand{\figcaption}[1]{\def\@captype{figure}\caption{#1}}
\newcommand{\tblcaption}[1]{\def\@captype{table}\caption{#1}}
\makeatother

\begin{document}
作成者: りーぜんと(Twitter: @50m\_regent)
\section*{問題1}
    \subsection*{(1)}
        $
            \displaystyle I_{1, k} = \int^\infty_0\frac{x}{(1+x^2)^{k+1}}dx = \left[\frac{-1}{2k(1+x^2)^k}\right]^{x=\infty}_{x=0} = \frac{1}{2k}
        $
    \subsection*{(2)}
        部分積分を行う。
        \begin{eqnarray}
            I_{n, k} &=& \int^\infty_0x^{2n - 2}\frac{x}{(1 + x^2)^{k + 1}}dx \nonumber \\
            &=& \left[x^{2n - 2}\frac{-1}{2k(1 + x ^ 2)^k}\right]^{x=\infty}_{x=0} + \frac{2n-2}{2k}\int^\infty_0\frac{x^{2(n - 1) - 1}}{(1 + x^2)^{(k - 1) + 1}}dx \nonumber \\
            &=& \frac{n - 1}{k}I_{n - 1, k - 1} \ \left(\because n - 1 - k < 0 \leftrightarrow \left[x^{2n - 2}\frac{-1}{2k(1 + x ^ 2)^k}\right]^{x=\infty}_{x=0} = 0 \right) \nonumber
        \end{eqnarray}
        \qed
    \subsection*{(3)}
        (2)より、
        $\displaystyle I_{n, k} = \frac{n - 1}{k}I_{n - 1, k - 1} = \frac{n - 1}{k}\frac{n - 2}{k - 1}I_{n - 2, k - 2} = \cdots = \frac{I_{1, {k - n + 1}}}{{}_k\mathrm C_{n - 1}} \ (\because n - 1 - k < 0)$

        (1)より、
        $\displaystyle I_{1, {k - n + 1}} = \frac{1}{2(k - n + 1)}$なので、
        $\displaystyle I_{n, k} = \frac{1}{2_k\mathrm C_{n - 1}(k - n + 1)}$
    \subsection*{(4)}
        与式を変形して、
        $\displaystyle\left(\frac{x^2}{x^2 + 1}\right)^{k + 1} \geq C_k$を得る。
        $x\geq 1$において$\displaystyle\frac{x^2}{x^2 + 1}$は単調増加で、
        $x = 1$のとき最小値$\displaystyle\frac{1}{2}$を取る。
        よって、$\displaystyle C_k=\frac{1}{2^{k + 1}}$とすれば与式が満たされる。
        \qed
    \subsection*{(5)}
        \begin{eqnarray}
            I_{n, k} &=& \lim_{L\rightarrow\infty}\int^L_0\frac{x^{2n - 1}}{(x^2 + 1)^{k + 1}}dx \nonumber \\
            &\geq& \int^\infty_0\frac{C_k}{x^{2k - 2n + 3}}dx \nonumber \\
            &=& \frac{C_k}{2(n - k - 1)}\left[x^{2(n - k - 1)}\right]^{x = \infty}_{x = 0} \nonumber \\
            &=& \infty \ (\because n - k - 1 \geq 0) \nonumber
        \end{eqnarray}
        \qed
\section*{問題2}
    \subsection*{(1)}
        $\det A = (a + 1)(a - 0.5) = 0$を満たせば良いので、$a = -1, 0.5$
    \subsection*{(2)}
        $a \neq -1, 0.5$のとき、$
            \displaystyle\begin{pmatrix}
                x_1 \\
                x_2 \\
            \end{pmatrix} =
            A^{-1}\begin{pmatrix}
                b_1 \\
                b_2 \\
            \end{pmatrix} =
            \frac{1}{(a + 1)(a - 0.5)}\begin{pmatrix}
                (a + 0.5)b_1 - 0.5b_2 \\
                -b_1 + ab_2 \\
            \end{pmatrix}
        $

        $a = -1$のとき、$
        \begin{pmatrix}
            -1 & 0.5 \\
            1 & -0.5 \\
        \end{pmatrix}\begin{pmatrix}
            x_1 \\
            x_2 \\
        \end{pmatrix} = \begin{pmatrix}
            -(x_1 - 0.5x_2) \\
            x_1 - 0.5x_2 \\
        \end{pmatrix} = \begin{pmatrix}
            b_1 \\
            b_2 \\
        \end{pmatrix}
        $より、$b_1 = -b_2$を満たさなければならない。
        このときの解は、
        \begin{equation*}
            \begin{pmatrix}
                x_1 \\
                x_2 \\
            \end{pmatrix} = \begin{pmatrix}
                b_2 \\
                0 \\
            \end{pmatrix} + C_1\begin{pmatrix}
                1 \\
                2 \\
            \end{pmatrix} \ (C_1は定数)
        \end{equation*}

        同様に$a = 0.5$のとき、$b_1 = 0.5b_2$を満たさなければならない。
        このときの解は、
        \begin{equation*}
            \begin{pmatrix}
                x_1 \\
                x_2 \\
            \end{pmatrix} = \begin{pmatrix}
                b_2 \\
                0 \\
            \end{pmatrix} + C_2\begin{pmatrix}
                -1 \\
                1 \\
            \end{pmatrix} \ (C_2は定数)
        \end{equation*}
    \subsection*{(3)}
        固有方程式は、
        $\det (A - \lambda E) = {\lambda - (a + 1)}{\lambda - (a - 0.5)} = 0$
        であるので、固有値は$a + 1, a - 0.5$

        固有ベクトルを$
        \begin{pmatrix}
            x \\
            y \\
        \end{pmatrix}
        $と表す。

        $\lambda = a + 1$について、
            \begin{equation*}
                A\begin{pmatrix}
                    x \\
                    y \\
                \end{pmatrix} =
                \begin{pmatrix}
                    a & 0.5 \\
                    1 & a + 0.5 \\
                \end{pmatrix}
                \begin{pmatrix}
                    x \\
                    y \\
                \end{pmatrix} =
                \lambda
                \begin{pmatrix}
                    x \\
                    y \\
                \end{pmatrix} =
                \begin{pmatrix}
                    (a + 1)x \\
                    (a + 1)y \\
                \end{pmatrix}
                \therefore
                \begin{pmatrix}
                    x \\
                    y \\
                \end{pmatrix} =
                C_1\begin{pmatrix}
                    1 \\
                    2 \\
                \end{pmatrix} \
                (C_1は定数)
            \end{equation*}
        
        $\lambda = a - 0.5$についても同様に、
        \begin{equation*}
            \begin{pmatrix}
                a & 0.5 \\
                1 & a + 0.5 \\
            \end{pmatrix}
            \begin{pmatrix}
                x \\
                y \\
            \end{pmatrix} =
            \begin{pmatrix}
                (a - 0.5)x \\
                (a - 0.5)y \\
            \end{pmatrix}
            \therefore
            \begin{pmatrix}
                x \\
                y \\
            \end{pmatrix} =
            C_2\begin{pmatrix}
                1 \\
                -1 \\
            \end{pmatrix} \
            (C_2は定数)
        \end{equation*}

        \begin{equation*}
            \therefore\begin{pmatrix}
                1 & 1 \\
                2 & -1 \\
            \end{pmatrix}^{-1}A\begin{pmatrix}
                1 & 1 \\
                2 & -1 \\
            \end{pmatrix} = \begin{pmatrix}
                a + 1 & 0 \\
                0 & a - 0.5 \\
            \end{pmatrix}
        \end{equation*}
    \subsection*{(4)}
        (3)より、
            \begin{eqnarray}
                A^n &=& \begin{pmatrix}
                    1 & 1 \\
                    2 & -1 \\
                \end{pmatrix}\begin{pmatrix}
                    a + 1 & 0 \\
                    0 & a - 0.5 \\
                \end{pmatrix}^n\begin{pmatrix}
                    1 & 1 \\
                    2 & -1 \\
                \end{pmatrix}^{-1} \nonumber \\
                &=& \frac{1}{3}\begin{pmatrix}
                    (a + 1)^n + 2(a - 0.5)^n & (a + 1)^n - (a - 0.5)^n \\
                    2(a + 1)^n - 2(a - 0.5)^n & 2(a + 1)^n + (a - 0.5)^n \\
                \end{pmatrix} \nonumber
            \end{eqnarray}
    \subsection*{(5)}
        $\displaystyle\lim_{n\rightarrow\infty}A^n = \begin{pmatrix}
            0 & 0 \\
            0 & 0 \\
        \end{pmatrix}$であればよいので、
        $(a + 1)^n = 0$かつ$(a - 0.5)^n = 0$
        を満たせば良い。
        \begin{equation*}
            \therefore |a + 1| < 0 \cap |a - 0.5| < 0 \leftrightarrow -0.5 < a < 0
        \end{equation*}
\section*{問題3}
    \subsection*{(1)}
        二桁の平方数を全て調べれば良い。$(a, b) = (0, 0), (0, 1), (8, 1)$
    \subsection*{(2)}
        \subsubsection*{(2-1)}
            コインの枚数が増減する確率はそれぞれ
            $\displaystyle\frac{45}{100} = \frac{9}{20}$、
            増減しない確率は$\displaystyle\frac{10}{100} = \frac{1}{10}$
            
            \begin{eqnarray}
                \therefore P_k &=& \frac{9}{20}P_{k - 1} + \frac{1}{10}P_k + \frac{9}{20}P_{k + 1} \nonumber \\
                \leftrightarrow 2P_k &=& P_{k - 1} + P_{k + 1} \nonumber
            \end{eqnarray}
        \subsubsection*{(2-2)}
            (2-1)より、$P_{k + 1} - P_k = P_k - P_{k - 1}$なので、$P_k$は等差数列となる。
            $P_0 = 0, P_{10} = 1$より、$\displaystyle P_k = \frac{k}{10}$
        \subsubsection*{(2-3)}
            得点の期待値は、$\displaystyle\frac{k(10 - k)^2}{10}$である。
            これの極値を考える。

            \begin{eqnarray}
                \frac{d}{dk}\frac{k(10 - k)^2}{10} &=& \frac{(10 - k)(10 - 3k)}{10} \nonumber \\
                \therefore \frac{d}{dk}\frac{k(10 - k)^2}{10} = 0 &\leftrightarrow& k = \frac{10}{3} \ (\because 1 \leq k \leq 9) \nonumber
            \end{eqnarray}

            これに最も近い整数は$k = 3$で、$k = 2, 4$
            を代入すると極大値を取ることがわかる。$\therefore k = 3$
\end{document}