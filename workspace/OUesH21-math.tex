\documentclass[]{jsarticle}
\usepackage[dvipdfmx]{graphicx}
\usepackage{bm}
\usepackage{amsmath}
\usepackage{amssymb}
\usepackage{amsfonts}
\usepackage{amsthm}
\usepackage{comment}
\usepackage{listings}
\usepackage{empheq}
\usepackage{siunitx}
\usepackage[hyphens]{url}
\lstset{
    basicstyle={\ttfamily},
    identifierstyle={\small},
    commentstyle={\smallitshape},
    keywordstyle={\small\bfseries},
    ndkeywordstyle={\small},
    stringstyle={\small\ttfamily},
    frame={tb},
    breaklines=true,
    columns=[l]{fullflexible},
    numbers=left,
    xrightmargin=0zw,
    xleftmargin=3zw,
    numberstyle={\scriptsize},
    stepnumber=1,
    numbersep=1zw,
    lineskip=-0.5ex,
    keepspaces=true,
    language=c
}
\renewcommand{\lstlistingname}{リスト}
\makeatletter
\newcommand{\figcaption}[1]{\def\@captype{figure}\caption{#1}}
\newcommand{\tblcaption}[1]{\def\@captype{table}\caption{#1}}
\makeatother

\begin{document}
作成者: りーぜんと(Twitter: @50m\_regent)
\section*{問題1}
    \subsection*{(1)}
        与式の両辺を$x$で微分する。
        \begin{equation*}
            2z\frac{\partial z}{\partial x} + z + (x + 2y)\frac{\partial z}{\partial x} - 4x = 0 \ \therefore \frac{\partial z}{\partial x} = \frac{4x - z}{x + 2y + 2z}
        \end{equation*}

        同様に与式の両辺を$y$で微分する。
        \begin{equation*}
            2z\frac{\partial z}{\partial y} + 2z + (x + 2y)\frac{\partial z}{\partial y} - 1 - y = 0 \ \therefore \frac{\partial z}{\partial y} = \frac{y - 2z + 1}{x + 2y + 2z}
        \end{equation*}
    \subsection*{(2)}
        停留点を求める。
        \begin{equation*}
            \frac{\partial z}{\partial x} = \frac{\partial z}{\partial y} = 0 \leftrightarrow 4x - z = y - 2z + 1 = 0 \ \therefore y = 8x - 1, z = 4x
        \end{equation*}

        これを満たす点で極値を取ることがわかっているので、与式に代入する。

        \begin{eqnarray}
            &&(4x)^2 + (x + 2(8x - 1))\cdot 4x - (4 + 2x^2+ 8x - 1 + \frac{(8x - 1)^2}{2}) = 0 \nonumber \\
            &\leftrightarrow& (x, y) = \left(\frac{4 + \sqrt{191}}{50}, \frac{-9 + 4\sqrt{191}}{25}\right), z = \frac{8 + 2\sqrt{191}}{25} \ (\because z \geq 0) \nonumber
        \end{eqnarray}
    \subsection*{(3)}
        $f(x,y) = x + y - 1 = 0$とおくと、$\displaystyle\frac{\partial f}{\partial x} = \frac{\partial f}{\partial y} = 1, y = 1 - x$

        $\displaystyle\frac{\partial z}{\partial f} = \frac{\displaystyle\frac{\partial z}{\partial x}}{\displaystyle\frac{\partial f}{\partial x}} = \frac{\displaystyle\frac{\partial z}{\partial y}}{\displaystyle\frac{\partial f}{\partial y}}$
        であるので、(1)より、
        $4x - z = y - 2z + 1 \leftrightarrow 4x - y + z - 1 = 0$

        $y = 1 - x$を代入すると、$z = 2 - 5x$を得る。
        与式に代入して、

        \begin{eqnarray}
            &&(2 - 5x)^2 + \{x + 2(1 - x)\}(2 - 5x) - (4 + 2x^2+ 1 - x + \frac{(1 - x)^2}{2}) = 0 \nonumber \\
            &\leftrightarrow& (x, y) = \left(\frac{1}{11}, \frac{10}{11}\right), z = \frac{17}{11} \ (\because z \geq 0) \nonumber
        \end{eqnarray}
\section*{問題2}
    \subsection*{(1)}
        $A = \begin{pmatrix}
            -2 & 3 & 2 \\
            -6 & 7 & 2 \\
            3 & -3 & 3 \\
        \end{pmatrix}$
    \subsection*{(2)}
        固有方程式は、$\det (A - \lambda E) = (1 - \lambda)(3 - \lambda)(4 - \lambda) = 0 \ \therefore \lambda = 1, 3, 4$

        固有ベクトルを$
        \begin{pmatrix}
            x \\
            y \\
            z \\
        \end{pmatrix}
        $と表す。

        $\lambda = 1$について、
            \begin{equation*}
                A\begin{pmatrix}
                    x \\
                    y \\
                    z \\
                \end{pmatrix} =
                \begin{pmatrix}
                    -2 & 3 & 2 \\
                    -6 & 7 & 2 \\
                    3 & -3 & 3 \\
                \end{pmatrix}
                \begin{pmatrix}
                    x \\
                    y \\
                    z \\
                \end{pmatrix} =
                \lambda
                \begin{pmatrix}
                    x \\
                    y \\
                    z \\
                \end{pmatrix} =
                \begin{pmatrix}
                    x \\
                    y \\
                    z \\
                \end{pmatrix}
                \therefore
                \begin{pmatrix}
                    x \\
                    y \\
                    z \\
                \end{pmatrix} =
                C_1\begin{pmatrix}
                    1 \\
                    1 \\
                    0 \\
                \end{pmatrix} \
                (C_1は定数)
            \end{equation*}

        $\lambda = 3$について、
            \begin{equation*}
                \begin{pmatrix}
                    -2 & 3 & 2 \\
                    -6 & 7 & 2 \\
                    3 & -3 & 3 \\
                \end{pmatrix}
                \begin{pmatrix}
                    x \\
                    y \\
                    z \\
                \end{pmatrix} =
                \begin{pmatrix}
                    3x \\
                    3y \\
                    3z \\
                \end{pmatrix}
                \therefore
                \begin{pmatrix}
                    x \\
                    y \\
                    z \\
                \end{pmatrix} =
                C_2\begin{pmatrix}
                    1 \\
                    1 \\
                    1 \\
                \end{pmatrix} \
                (C_2は定数)
            \end{equation*}

        $\lambda = 4$について、
            \begin{equation*}
                \begin{pmatrix}
                    -2 & 3 & 2 \\
                    -6 & 7 & 2 \\
                    3 & -3 & 3 \\
                \end{pmatrix}
                \begin{pmatrix}
                    x \\
                    y \\
                    z \\
                \end{pmatrix} =
                \begin{pmatrix}
                    4x \\
                    4y \\
                    4z \\
                \end{pmatrix}
                \therefore
                \begin{pmatrix}
                    x \\
                    y \\
                    z \\
                \end{pmatrix} =
                C_3\begin{pmatrix}
                    1 \\
                    0 \\
                    3 \\
                \end{pmatrix} \
                (C_3は定数)
            \end{equation*}
    \subsection*{(3)}
        (2)より、$
            P = \begin{pmatrix}
                1 & 1 & 1 \\
                1 & 1 & 0 \\
                0 & 1 & 3 \\
            \end{pmatrix}, P^{-1}AP = \begin{pmatrix}
                1 & 0 & 0 \\
                0 & 3 & 0 \\
                0 & 0 & 4 \\
            \end{pmatrix}
        $
    \subsection*{(4)}
        $\mathbf{x}_n = A^{n - 1}\mathbf{x}_1$である。$A^{n - 1}$を求める。
        (3)より、$
            P^{-1}AP = \begin{pmatrix}
                1 & 0 & 0 \\
                0 & 3 & 0 \\
                0 & 0 & 4 \\
            \end{pmatrix} \leftrightarrow A = P\begin{pmatrix}
                1 & 0 & 0 \\
                0 & 3 & 0 \\
                0 & 0 & 4 \\
            \end{pmatrix}P^{-1}
        $なので、

    \begin{equation*}
        A^{n - 1} = P\begin{pmatrix}
            1 & 0 & 0 \\
            0 & 3 & 0 \\
            0 & 0 & 4 \\
        \end{pmatrix}^nP^{-1} = \begin{pmatrix}
            3 - 3^n + 4^{n - 1} & -2 + 3^n - 4 ^{n - 1} & -1 + 3^{n - 1} \\
            3 - 3^n & -2 + 3^n & -1 + 3^{n - 1} \\
            -3^n + 3\cdot 4^{n - 1} & 3^n - 3\cdot 4^{n - 1} & 3^{n - 1} \\
        \end{pmatrix}
    \end{equation*}

    よって、$
        \begin{pmatrix}
            x_n \\
            y_n \\
            z_n \\
        \end{pmatrix} = \begin{pmatrix}
            3 - 3^n + 4^{n - 1} & -2 + 3^n - 4 ^{n - 1} & -1 + 3^{n - 1} \\
            3 - 3^n & -2 + 3^n & -1 + 3^{n - 1} \\
            -3^n + 3\cdot 4^{n - 1} & 3^n - 3\cdot 4^{n - 1} & 3^{n - 1} \\
        \end{pmatrix}\begin{pmatrix}
            1 \\
            1 \\
            2 \\
        \end{pmatrix} = \begin{pmatrix}
            -1 + 2\cdot 3^{n - 1} \\
            -1 + 2\cdot 3^{n - 1} \\
            2\cdot 3^{n - 1} \\
        \end{pmatrix}
    $
\section*{問題3}
    \subsection*{(1)}
        $P(A) = P(A\cap C) + P(A\cap \bar{C}) = P(A|C)P(C) +P(A|\bar{C})P(\bar{C})$
        \qed
    \subsection*{(2)}
        (1)における$A$を$A\cap B$とすると、

        \begin{eqnarray}
            P(A\cap B) &=& P(A\cap B|C)P(C) +P(A\cap B|\bar{C})P(\bar{C}) \nonumber \\
            &=& P(A|C)P(B|C)P(C) + P(A|\bar{C})P(B|\bar{C})P(\bar{C}) \nonumber
        \end{eqnarray}
        \qed
    \subsection*{(3)}
        (1)の式に$P(A|C) = P(A|\bar{C})$を代入すると、
        $P(A) = P(A|C)(P(C) +P(\bar{C})) = P(A|C)$
        となり、$A$と$C$は独立である。
        \qed
    \subsection*{(4)}
        (1)より、
        $\displaystyle P(A|\bar{C}) = \frac{P(A) -P(A|C)P(C)}{P(\bar{C})}, P(B|\bar{C}) = \frac{P(B) -P(B|C)P(C)}{P(\bar{C})}$

        (2)の式と、$A$と$B$が独立であることより、

        \begin{eqnarray}
            P(A\cap B) &=& P(A)P(B) = P(A|C)P(B|C)P(C) + P(A|\bar{C})P(B|\bar{C})P(\bar{C}) \nonumber \\
            &=& P(A|C)P(B|C)P(C) + \frac{\{P(A) -P(A|C)P(C)\}\{P(B) -P(B|C)P(C)\}}{P(\bar{C})} \nonumber \\
            \leftrightarrow - P(A)P(B)P(C) &=& \{P(A|C)P(B|C) - P(A)P(B|C) - P(A|C)P(B)\}P(C) + P(A|C)P(B|C)P(C)^2 \nonumber \\
            &\leftrightarrow& \{P(A) - P(A|C)\}\{P(B) - P(B|C)\} = 0 \nonumber \\
            &\leftrightarrow& P(A) = P(A|C) \vee P(B) = P(B|C) \nonumber
        \end{eqnarray}

        これより、$A$と$C$が独立であるか、または$B$と$C$が独立であることがわかる。
        \qed
\end{document}