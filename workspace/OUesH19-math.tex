\documentclass[]{jsarticle}
\usepackage[dvipdfmx]{graphicx}
\usepackage{bm}
\usepackage{amsmath}
\usepackage{amssymb}
\usepackage{amsfonts}
\usepackage{amsthm}
\usepackage{comment}
\usepackage{listings}
\usepackage{cases}
\usepackage{siunitx}
\usepackage[hyphens]{url}
\lstset{
    basicstyle={\ttfamily},
    identifierstyle={\small},
    commentstyle={\smallitshape},
    keywordstyle={\small\bfseries},
    ndkeywordstyle={\small},
    stringstyle={\small\ttfamily},
    frame={tb},
    breaklines=true,
    columns=[l]{fullflexible},
    numbers=left,
    xrightmargin=0zw,
    xleftmargin=3zw,
    numberstyle={\scriptsize},
    stepnumber=1,
    numbersep=1zw,
    lineskip=-0.5ex,
    keepspaces=true,
    language=c
}
\renewcommand{\lstlistingname}{リスト}
\makeatletter
\newcommand{\figcaption}[1]{\def\@captype{figure}\caption{#1}}
\newcommand{\tblcaption}[1]{\def\@captype{table}\caption{#1}}
\makeatother

\begin{document}
作成者: りーぜんと(Twitter: @50m\_regent)
\section*{問題1}
    \subsection*{(1)}
        $
            \displaystyle\frac{\partial (x, y, z)}{\partial (r, \theta, \varphi)} =
            \begin{vmatrix}
                \displaystyle\frac{\partial x}{\partial r} & \displaystyle\frac{\partial x}{\partial \theta} & \displaystyle\frac{\partial x}{\partial \varphi} \\
                \displaystyle\frac{\partial y}{\partial r} & \displaystyle\frac{\partial y}{\partial \theta} & \displaystyle\frac{\partial y}{\partial \varphi} \\
                \displaystyle\frac{\partial z}{\partial r} & \displaystyle\frac{\partial z}{\partial \theta} & \displaystyle\frac{\partial z}{\partial \varphi} \\
            \end{vmatrix} =
            \begin{vmatrix}
                \sin\theta\cos\varphi & r\cos\theta\cos\varphi & -r\sin\theta\sin\varphi \\
                \sin\theta\sin\varphi & r\cos\theta\sin\varphi & r\sin\theta\cos\varphi \\
                \cos\theta & -r\sin\theta & 0 \\
            \end{vmatrix} = r^2\sin\theta
        $
    \subsection*{(2)}
        \begin{equation*}
            I\left(\frac{1}{2}\right) =
            \int^\infty_{-\infty}\int^\infty_{-\infty}\int^\infty_{-\infty}\frac{e^{-(x^2 + y^2 + z^2)}}{\sqrt{x^2 + y^2 + z^2}}dxdydz
        \end{equation*}

        (1)の変数変換を施す。
        \begin{equation*}
            x^2 + y^2 + z^2 = r^2, dxdydz=r^2\sin\theta drd\theta d\varphi
        \end{equation*}
        \begin{equation*}
            \therefore I\left(\frac{1}{2}\right) =
            \int^{2\pi}_0\int^\pi_0\int^\infty_0r\sin\theta e^{-r^2}drd\theta d\varphi =
            \int^{2\pi}_0d\varphi\int^\pi_0\sin\theta d\theta\int^\infty_0re^{-r^2}dr =
            2\pi
        \end{equation*}
    \subsection*{(3)}
        (2)と同様に変数変換を施すと、
        $\displaystyle I(\alpha) = 4\pi\int^\infty_0\frac{r^{2(1-\alpha)}}{e^{r^2}}dr$
        を得る。
        $r^2=t$として変数変換を行う。
        \begin{eqnarray}
            I(\alpha) &=& 4\pi\int^\infty_0\frac{t^{1-\alpha}}{e^t}\frac{dt}{2\sqrt{t}} \ \left(\because dr = \frac{dt}{2\sqrt{t}}\right) \nonumber \\
            &=& 2\pi\int^\infty_0\frac{t^{\frac{1}{2}-\alpha}}{e^t}dt \nonumber \\
            &=& \frac{2\pi}{\frac{3}{2} - \alpha}\int^\infty_0\frac{t^{\frac{3}{2}-\alpha}}{e^t}dt \ (部分積分) \nonumber \\
            &=& \frac{1}{\frac{3}{2} - \alpha}2\pi\int^\infty_0\frac{t^{\frac{1}{2}-(\alpha - 1)}}{e^t}dt \nonumber \\
            &=& \frac{I(\alpha - 1)}{\frac{3}{2} - \alpha} \nonumber
        \end{eqnarray}

        $\displaystyle\frac{3}{2} - \alpha > 0$であればこれは収束するので、
        求める範囲は、$\displaystyle\frac{3}{2} > \alpha$
    \subsection*{(4)}
        (2)と同様に変数変換を施すと、
        $\displaystyle J(\alpha, \beta) = 2^{2 - \beta}\pi\int^\frac{1}{2}_0\frac{r^{2(1-\alpha)}}{|\log r|^\beta}dr$
        を得る。
        $\log r = t$として変数変換を行う。
        \begin{eqnarray}
            J(\alpha, \beta) &=& 2^{2 - \beta}\pi\int^{-\log 2}_{-\infty}\frac{e^{2(1-\alpha)t}}{|t|^\beta}e^tdt \ \left(\because dr = e^tdt\right) \nonumber \\
            &=& 2^{2 - \beta}\pi\int^{-\log 2}_{-\infty}\frac{e^{2(\frac{3}{2}-\alpha)t}}{(-t)^\beta}dt \nonumber
        \end{eqnarray}
        $t < 0$であることに注意すると、これも(3)と同様に
        $\displaystyle\frac{3}{2} - \alpha > 0$で収束することがわかるので、
        求める条件は、$\displaystyle\frac{3}{2} > \alpha, -\infty<\beta<\infty$
\section*{問題2}
    \subsection*{(1)}
        $\det A = a^3+ a^2 + 1 - a - a - a^3 = (a - 1)^2$
    \subsection*{(2)}
        $\mathrm{rank} \ A = 3 \leftrightarrow \det A \neq 0$を満たせば良いので、(1)より、$a \neq 1$
    \subsection*{(3)}
        固有方程式は、
        \begin{equation}
            \det (A - \lambda E) = (a - \lambda)\{\lambda^2 -(2a + 1)\lambda + a - 1\} = 0 \label{equ:char}
        \end{equation}
        である。
        $\lambda = 0$がこれを満たすので、(1)より、$\det (A - \lambda E) = \det A = 0 \leftrightarrow a = 1$となる。
        固有方程式に代入すると、$\lambda = 0, 3$を得る。

        固有ベクトルを$
        \begin{pmatrix}
            x \\
            y \\
            z \\
        \end{pmatrix}
        $と表す。

        $\lambda = 0$について、
            \begin{equation*}
                A
                \begin{pmatrix}
                    x \\
                    y \\
                    z \\
                \end{pmatrix} =
                \begin{pmatrix}
                    1 & 1 & 1 \\
                    1 & 1 & 1 \\
                    1 & 1 & 1 \\
                \end{pmatrix}
                \begin{pmatrix}
                    x \\
                    y \\
                    z \\
                \end{pmatrix} =
                \lambda
                \begin{pmatrix}
                    x \\
                    y \\
                    z \\
                \end{pmatrix} =
                \begin{pmatrix}
                    0 \\
                    0 \\
                    0 \\
                \end{pmatrix}
                \therefore
                \begin{pmatrix}
                    x \\
                    y \\
                    z \\
                \end{pmatrix} =
                C_1\begin{pmatrix}
                    1 \\
                    -1 \\
                    0 \\
                \end{pmatrix} +
                C_2\begin{pmatrix}
                    1 \\
                    0 \\
                    -1 \\
                \end{pmatrix} \
                (C_1, C_2は定数)
            \end{equation*}
        
        $\lambda = 3$についても同様に、
        \begin{equation*}
            \begin{pmatrix}
                1 & 1 & 1 \\
                1 & 1 & 1 \\
                1 & 1 & 1 \\
            \end{pmatrix}
            \begin{pmatrix}
                x \\
                y \\
                z \\
            \end{pmatrix} =
            \begin{pmatrix}
                3x \\
                3y \\
                3z \\
            \end{pmatrix}
            \therefore
            \begin{pmatrix}
                x \\
                y \\
                z \\
            \end{pmatrix} =
            C_3\begin{pmatrix}
                1 \\
                -1 \\
                -1 \\
            \end{pmatrix} \
            (C_3は定数)
        \end{equation*}
    \subsection*{(4)}
        式\ref{equ:char}に$\lambda = 1$を代入すると、$a = -1$を得る。($\because a < 0$)
        同式にこれを代入すると、$\lambda = 1, -2$を得る。

        (3)と同様に固有ベクトルを求める。

        $\lambda = 1$について、
        \begin{equation*}
            \begin{pmatrix}
                -1 & 1 & 1 \\
                1 & -1 & 1 \\
                1 & 1 & -1 \\
            \end{pmatrix}
            \begin{pmatrix}
                x \\
                y \\
                z \\
            \end{pmatrix} =
            \begin{pmatrix}
                x \\
                y \\
                z \\
            \end{pmatrix}
            \therefore
            \begin{pmatrix}
                x \\
                y \\
                z \\
            \end{pmatrix} =
            C_1\begin{pmatrix}
                1 \\
                1 \\
                1 \\
            \end{pmatrix} \
            (C_1は定数)
        \end{equation*}

        $\lambda = -2$について、
        \begin{equation*}
            \begin{pmatrix}
                -1 & 1 & 1 \\
                1 & -1 & 1 \\
                1 & 1 & -1 \\
            \end{pmatrix}
            \begin{pmatrix}
                x \\
                y \\
                z \\
            \end{pmatrix} =
            \begin{pmatrix}
                -2x \\
                -2y \\
                -2z \\
            \end{pmatrix}
            \therefore
            \begin{pmatrix}
                x \\
                y \\
                z \\
            \end{pmatrix} =
            C_2\begin{pmatrix}
                1 \\
                -1 \\
                0 \\
            \end{pmatrix} +
            C_3\begin{pmatrix}
                1 \\
                0 \\
                -1 \\
            \end{pmatrix} \
            (C_2, C_3は定数)
        \end{equation*}

        よって、
        \begin{eqnarray}
            &&\begin{pmatrix}
                1 & 1 & 1 \\
                1 & -1 & 0 \\
                1 & 0 & -1 \\
            \end{pmatrix}
            \begin{pmatrix}
                -1 & 1 & 1 \\
                1 & -1 & 1 \\
                1 & 1 & -1 \\
            \end{pmatrix}
            \begin{pmatrix}
                1 & 1 & 1 \\
                1 & -1 & 0 \\
                1 & 0 & -1 \\
            \end{pmatrix}^{-1} =
            \begin{pmatrix}
                1 & 0 & 0 \\
                0 & -2 & 0 \\
                0 & 0 & -2 \\
            \end{pmatrix} \nonumber \\
            &\leftrightarrow& A = \begin{pmatrix}
                -1 & 1 & 1 \\
                1 & -1 & 1 \\
                1 & 1 & -1 \\
            \end{pmatrix} =
            \begin{pmatrix}
                1 & 1 & 1 \\
                1 & -1 & 0 \\
                1 & 0 & -1 \\
            \end{pmatrix}^{-1}
            \begin{pmatrix}
                1 & 0 & 0 \\
                0 & -2 & 0 \\
                0 & 0 & -2 \\
            \end{pmatrix}
            \begin{pmatrix}
                1 & 1 & 1 \\
                1 & -1 & 0 \\
                1 & 0 & -1 \\
            \end{pmatrix} \nonumber \\
            &\leftrightarrow& A^n = \begin{pmatrix}
                1 & 1 & 1 \\
                1 & -1 & 0 \\
                1 & 0 & -1 \\
            \end{pmatrix}^{-1}
            \begin{pmatrix}
                1 & 0 & 0 \\
                0 & -2 & 0 \\
                0 & 0 & -2 \\
            \end{pmatrix}^n
            \begin{pmatrix}
                1 & 1 & 1 \\
                1 & -1 & 0 \\
                1 & 0 & -1 \\
            \end{pmatrix} \nonumber \\
            && \ \ \ \ \ = \frac{1}{3}\begin{pmatrix}
                1 - (-2)^{n + 1} & 1 - (-2)^n & 1 - (-2)^n \\
                1 - (-2)^n & 1 - (-2)^{n + 1} & 1 - (-2)^n \\
                1 - (-2)^n & 1 - (-2)^n & 1 - (-2)^{n + 1} \\
            \end{pmatrix} \nonumber
        \end{eqnarray}
\section*{問題3}
    \subsection*{(1)}
        $\displaystyle\frac{1}{n}$
    \subsection*{(2)}
        $\displaystyle\frac{1}{n}\cdot\frac{1}{n - 1}\cdot\cdots\cdot\frac{1}{n - m + 1} = \frac{(n - m)!}{n!}$
    \subsection*{(3)}
        $N = 1$のとき式($i$)が成り立ことは自明。

        $N = t - 1$のときに式が成り立っているとする。$N = t$のとき、
        \begin{eqnarray}
            P(A_1\cup A_2\cup\cdots\cup A_t)
            &=& P(A_1\cup A_2\cup\cdots\cup A_{t - 1}) + P(A_t) - P((A_1\cup A_2\cup\cdots\cup A_{t - 1}) \cap A_t) \nonumber \\
            &=& \sum^{t - 1}_{l = 1}(-1)^{l - 1}S_l + P(A_t) + \nonumber \\
            && (-1)^1\sum_{k_1}P(A_{k_1}\cap A_t) + (-1)^2\sum_{k_1 < k_2}P(A_{k_1}\cap A_{k_2}\cap A_t) + \cdots \nonumber \\
            && + (-1)^{t - 1}\sum_{k_1 < k_2 < \cdots < k_{t - 1}}P(A_{k_1}\cap A_{k_2}\cap\cdots\cap A_{k_{t - 1}} \cap A_t) \nonumber \\
            &=& \sum^{t}_{l = 1}(-1)^{l - 1}S_l \nonumber
        \end{eqnarray}
        \qed
    \subsection*{(4)}
        式($i$)より、$\displaystyle Q(1, n) = P(M_1\cup M_2\cup\cdots\cup M_n) = \sum^{n}_{j = 1}(-1)^{j - 1}S_j$である。
        $\displaystyle S_j = {}_n\mathrm C_j\frac{(n - j)!}{n!} = \frac{1}{j!}$であるので、
        \begin{equation*}
            Q(1, n)= \sum^{n}_{j = 1}(-1)^{j - 1}\frac{1}{j!}
        \end{equation*}
        \qed
    \subsection*{(5)}
        $\displaystyle\lim_{n\rightarrow\infty}Q(1, n) = -\sum^\infty_{j = 0}\frac{(-1)^j}{j!} + 1 = 1 - e^{-1} = 1 - \frac{1}{e}$
\end{document}